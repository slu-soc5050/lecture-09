\documentclass{tufte-handout}

\usepackage{xcolor}

% set hyperlink attributes
\hypersetup{colorlinks}

% set image attributes:
\usepackage{graphicx}
\graphicspath{ {images/} }

% indent subsections
\newenvironment{subs}
  {\adjustwidth{3em}{0pt}}
  {\endadjustwidth}

% ============================================================

% define the title
\title{SOC 4015/5050: Lecture 09 Functions}
\author{Christopher Prener, Ph.D.}
\date{Fall 2018}
% ============================================================
\begin{document}
% ============================================================
\maketitle % generates the title
% ============================================================

\vspace{5mm}
\section{Packages}
\begin{itemize}
\item \texttt{base}
\item \texttt{janitor}
\item \texttt{tidyverse}
\begin{itemize}
\item \texttt{dplyr}
\item \texttt{forcats}
\end{itemize}
\end{itemize}

\vspace{5mm}
\section{Missing Data with \texttt{ifelse()}}
\noindent \texttt{dplyr::}{\color{red}\texttt{mutate}}\texttt{(\textit{dataFrame}, \textit{var} = base::}{\color{red}\texttt{ifelse}}\texttt{(\textit{exp}, NA, \textit{var}))}

\vspace{5mm}
\section{Recoding Data with \texttt{case\_when()}}
\noindent \texttt{dplyr::}{\color{red}\texttt{mutate}}\texttt{(\textit{dataFrame}, \textit{var} = dplyr::}{\color{red}\texttt{case\_when}}\texttt{( \\ \textit{var} == \textit{val} \textasciitilde{} \textit{newVal}, \\ \textit{var} == \textit{val} \textasciitilde{} \textit{newVal}))}

\vspace{5mm}
\section{Convert to Factor}
\noindent \texttt{base::}{\color{red}\texttt{as.factor}}\texttt{(\textit{var})}

\vspace{5mm}
\section{Releveling Factor}
\noindent \texttt{dplyr::}{\color{red}\texttt{mutate}}\texttt{(\textit{dataFrame}, \textit{var} = dplyr::}{\color{red}\texttt{fct\_relevel}}\texttt{(\textit{var}, \\ "\textit{level}", "\textit{level}")}

\vspace{5mm}
\section{Recoding Factor}
\noindent \texttt{dplyr::}{\color{red}\texttt{mutate}}\texttt{(\textit{dataFrame}, \textit{var} = dplyr::}{\color{red}\texttt{fct\_recode}}\texttt{(\textit{var}, \\ "\textit{newLevel}" = "\textit{oldLevel}", \\ "\textit{newLevel}" = "\textit{oldLevel}"))}

% ============================================================
\end{document}